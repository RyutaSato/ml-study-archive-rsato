\section{データセット}
\label{sec:dataset}
本実験で使用する各データセットの詳細について述べる.
なお,クラス数はデータセットによって異なるため,2値分類問題は,データ数の多いクラスを'majority'クラスとして,他方を'minority'クラスとして集計する.
多クラスのデータセットにおいて,どのクラスを'minority'クラスとして評価するか,マクロ指標の定義については,別途定める.

\subsection{KDD Cup 1999 Data 10\%}
\begin{description}
    \item[概要] 知識発見とデータマイニングのコンテストであるKDD Cupで使用された不正侵入検知タスクのためのデータセットの10\%抽出データ.
    \item[発表年] 1999
    \item[サンプル数] 494,021
    \item[クラスごとのサンプル数] \mbox{}
        \begin{table}
            \centering
            \caption{KDD Cup 1999 Data 10\%のクラスごとのサンプル数}
                \label{tab:kddcup1999data10percent}
                \begin{tabular}{lrc} \hline
                    \multicolumn{1}{c}{クラス}&
                    \multicolumn{1}{c}{サンプル数}&
                    \multicolumn{1}{c}{割合}\\
                    \hline
                    \hline
                    normal  (majority)& 97,278 & 19.7\% \\
                    dos & 391,458 & 79.2\% \\
                    probe & 4,628 & 0.9\% \\
                    r2l & 1,605 & 0.3\% \\
                    u2r (minority)& 52 & 0.01\% \\ 
                    \hline
                \end{tabular}
        \end{table}
    \item[特徴量の数] 38 (カテゴリデータを除く)
    \item[各特徴量の種類] 正常アクセスと大きく4つに分類できる攻撃アクセスに関するインターネットトラフィックのデータが含まれている.
    \item[留意事項] このデータセットに含まれる3つのカテゴリデータは使用しない.
                   また,少数派クラスの評価にはu2rクラスのみを使用し,'majority'クラスの評価には'normal'クラスのみを使用する.全体のmacro F-accuracyは,5クラス全体で評価する.
\end{description}

\subsection{KDD CUP 1999 Data 10\% dropped}
\begin{description}
    \item[概要] KDD CUP 1999 Data 10\%から, 別の研究\cite{thesis1}により不要と判断された13個の特徴量とカテゴリ特徴量を除いたオリジナルのデータセット.
    \item[サンプル数] 同上
    \item[クラスごとのサンプル数] 同上
    \item[特徴量の数] 25 
\end{description}

\subsection{Credit Card Fraud Detection Dataset}
\begin{description}
    \item[概要] 金融取引データに基づくクレジットカード詐欺検出のためのデータセット.このデータセットは,正常な取引と不正な取引の2つのクラスに分類されたクレジットカード取引の情報を含んでいる.
    \item[提供元] Worldline and the Université Libre de Bruxelles (ULB)\cite{CreditCardFraudDetectionDataset}
    \item[サンプル数] 284,807
    \item[クラスごとのサンプル数と割合] \mbox{}
        \begin{table}
            \centering
            \caption{Credit Card Fraud Detection Datasetのクラスごとのサンプル数}
                \label{tab:creditcardfrauddetectiondataset}
                \begin{tabular}{lrc} \hline
                    \multicolumn{1}{c}{クラス}&
                    \multicolumn{1}{c}{サンプル数}&
                    \multicolumn{1}{c}{割合}\\
                    \hline
                    \hline
                    normal  (majority)& 284,315 & 99.83\% \\
                    fraud (minority)& 492 & 0.17\% \\
                    \hline
                \end{tabular}
        \end{table}
    \item[特徴量の数] 特徴量の総数(例: 30)
    \item[各特徴量の種類] \mbox{}
            V1, V2, ..., V28(PCAによる匿名化された特徴量)、取引金額('Amount')、取引後の時間('Time')
    \item[留意事項] このデータセットには,主に数値特徴量が含まれており,PCAによって特徴量が匿名化されている.
\end{description}

\subsection{ecoli}
\begin{description}
    \item[概要] 大腸菌のタンパク質が細胞内のどの部分に局在するかを示すデータセット. \cite{ecoli}
    \item[提供元] UCI Machine Learning Repository
    \item[サンプル数] 336
    \item[クラスごとのサンプル数と割合] \mbox{}
        \begin{table}
            \centering
            \caption{ecoliのクラスごとのサンプル数}
            \label{tab:ecoli}
            \begin{tabular}{lrc} \hline
                \multicolumn{1}{c}{クラス}&
                \multicolumn{1}{c}{サンプル数}&
                \multicolumn{1}{c}{割合}\\
                \hline
                \hline
                cp (majority)& 143 & 42.560\% \\
                im (majority)& 77 & 22.917\% \\
                imS (majority)& 2 & 0.595\% \\
                imL (majority)& 2 & 0.595\% \\
                imU (minority)& 35 & 10.417\% \\
                om (majority)& 20 & 5.952\% \\
                omL (majority)& 5 & 1.488\% \\
                pp (majority)& 52 & 15.476\% \\
                \hline
            \end{tabular}
        \end{table}
    \item[特徴量の数] 7
    \item[各特徴量の種類] 特徴量は,分析により判明した科学的特性を示したもの.\\
            mcg, gvh, lip, chg, aac, alm1, alm2
    \item[留意事項] 提供元により,'imU'クラスとそれ以外のクラスに2値化されている.
\end{description}

\subsection{optical\_digits}
    \begin{description}
    \item[概要] 手書きの数字の光学的認識を各ピクセルの値に基づいて示したデータセット.\cite{opticaldigits}
    \item[提供元] UCI Machine Learning Repository
    \item[サンプル数] 5,620
    \item[クラスごとのサンプル数と割合] majority: 5066 (90.1\%), minority: 554 (9.8\%) (詳細は表\ref{tab:opticaldigits}を参照)
        \begin{table}
            \centering
            \caption{optical\_digitsのクラスごとのサンプル数}
            \label{tab:opticaldigits}
            \begin{tabular}{lrc} \hline
                \multicolumn{1}{c}{クラス}&
                \multicolumn{1}{c}{サンプル数}&
                \multicolumn{1}{c}{割合}\\
                \hline
                \hline
                0 (minority)& 514 & 9.1\% \\
                1 (majority)& 571 & 10.2\% \\
                2 (majority)& 557 & 9.9\% \\
                3 (majority)& 572 & 10.2\% \\
                4 (majority)& 568 & 10.1\% \\
                5 (majority)& 558 & 9.9\% \\
                6 (majority)& 558 & 9.9\% \\
                7 (majority)& 566 & 10.1\% \\
                8 (minority)& 554 & 9.8\% \\
                9 (majority)& 562 & 10.0\% \\
                \hline
            \end{tabular}
        \end{table}
    \item[特徴量の数] 64
    \item[各特徴量の種類] \mbox{}
        各特徴量は8$\times$8ピクセルのグリッドに基づく手書き数字の光学的特性を0から16の整数値で表したもの.
    \item[留意事項] 提供元により,'8'クラスとそれ以外のクラスに2値化されている.
    \end{description}

    \subsection{satimage}
    \begin{description}
        \item[概要] 衛星画像データを基にした土地利用分類のためのデータセット.\cite{satimage}
        \item[提供元] UCI Machine Learning Repository
        \item[サンプル数] 6,435
        \item[クラスごとのサンプル数と割合] majority: 5,109 (79.4\%), minority: 1,326 (20.6\%) (詳細は表\ref{tab:satimage}を参照)
            
            \begin{table}
                \centering
                \caption{satimageのクラスごとのサンプル数}
                \label{tab:satimage}
                \begin{tabular}{lrc} \hline
                    \multicolumn{1}{c}{クラス}&
                    \multicolumn{1}{c}{サンプル数}&
                    \multicolumn{1}{c}{割合}\\
                    \hline
                    \hline
                    赤い土壌 (majority)& 1,533 & 23.8\% \\
                    綿作物 (majority)& 703 & 10.9\% \\
                    灰色の土壌 (majority)& 1,358 & 21.1\% \\
                    湿った灰色の土壌 (minority)& 626 & 9.7\% \\
                    切り株のある土壌 (majority)& 707 & 11.0\% \\
                    非常に湿った灰色の土壌 (majority)& 1,508 & 23.5\% \\
                    \hline
                \end{tabular}
            \end{table}
        \item[特徴量の数] 36
        \item[各特徴量の種類] \mbox{}
            各特徴量は,3$\times$3のピクセルそれぞれの4つのスペクトル (赤色,緑色,2種類の赤外線)の強さを0~255の値に変換したもの.
        \item[留意事項]  提供元により,'湿った灰色の土壌'クラスとそれ以外のクラスに2値化されている.
    \end{description}

\subsection{pen\_digits}
\begin{description}
    \item[概要] 44人の人物によって書かれた0から9までの数字のデータセット.\cite{pendigits}
    \item[提供元] Alpaydin, Ethem and Alimoglu, Fikret
    \item[サンプル数] 10,992
    \item[クラスごとのサンプル数と割合] majority: 9,937 (90.4\%), minority: 1,055 (9.6\%) (詳細は表\ref{tab:pendigits}を参照)
        \begin{table}
            \centering
            \caption{pen\_digitsのクラスごとのサンプル数}
            \label{tab:pendigits}
            \begin{tabular}{lrc} \hline
                \multicolumn{1}{c}{クラス}&
                \multicolumn{1}{c}{サンプル数}&
                \multicolumn{1}{c}{割合}\\
                \hline
                \hline
                0 (majority)& 1,143 & 9.6\% \\
                1 (majority)& 1,143 & 9.6\% \\
                2 (majority)& 1,144 & 9.6\% \\
                3 (majority)& 1,055 & 9.6\% \\
                4 (majority)& 1,144 & 9.6\% \\
                5 (minority)& 1,055 & 9.6\% \\
                6 (majority)& 1,056 & 9.6\% \\
                7 (majority)& 1,142 & 9.5\% \\
                8 (majority)& 1,055 & 9.6\% \\
                9 (majority)& 1,055 & 9.6\% \\
                \hline
            \end{tabular}
        \end{table}
    \item[特徴量の数] 16
    \item[各特徴量の種類] \mbox{}
        2次元座標上の描画における各点の座標を100msごとにサンプリングしたもののを等間隔に8点でサンプリングしたもののx座標とy座標の値.
    \item[留意事項]  提供元により,'5'クラスとそれ以外のクラスに2値化されている.
\end{description}

\subsection{shuttle}
\begin{description}
    \item[概要] NASAの宇宙シャトルの軌道運行中の観測データに基づくデータセット.\cite{shuttle}
    \item[提供元] UCI Machine Learning Repository
    \item[サンプル数] 58,000
    \item[クラスごとのサンプル数と割合] majority: 46,400 (80\%), minority: 11,600 (20\%) (詳細は表\ref{tab:shuttle}を参照)

        \begin{table}
            \centering
            \caption{shuttleのクラスごとのサンプル数}
            \label{tab:shuttle}
            \begin{tabular}{lrc} \hline
                \multicolumn{1}{c}{クラス}&
                \multicolumn{1}{c}{サンプル数}&
                \multicolumn{1}{c}{割合}\\
                \hline
                \hline
                クラス1 (majority)& 46,400 & 80\% \\
                クラス2 (minority)& ?? & ??\% \\
                ... & ... & ... \\
                クラス9 (minority)& ?? & ??\% \\
                \hline
            \end{tabular}
        \end{table}

    \item[特徴量の数] 9
    \item[各特徴量の種類] \mbox{}
        各特徴量はシャトルの軌道運行中の様々なセンサーからのデータを表しており、例えば温度、圧力、速度などが含まれる.
    \item[留意事項] 提供元により,クラス1とそれ以外のクラスに2値化されている.
\end{description}


\subsection{abalone および abalone\_19}
\begin{description}
    \item[概要] 物理測定値から,アワビの年齢を推定するデータセット \cite{abalone}
    \item[提供元] Nash,Warwick, Sellers,Tracy, Talbot,Simon, Cawthorn,Andrew, and Ford,Wes
    \item[サンプル数] 4,177
    \item[クラスごとのサンプル数と割合(abalone)] majority(クラス:7以外): 3786 (90.6\%), minority(クラス:7): 391 (9.4\%) (詳細は表\ref{tab:abalone}を参照)
    \item[クラスごとのサンプル数と割合(abalone\_19)] majority(クラス:19以外): 3786 (90.6\%), minority(クラス:19): 391 (9.4\%) (詳細は表\ref{tab:abalone}を参照)

    \begin{table}
        \centering
        \caption{abaloneのクラスごとのサンプル数}
        \label{tab:abalone}
        \begin{tabular}{lrc} \hline
            \multicolumn{1}{c}{クラス} &
            \multicolumn{1}{c}{サンプル数} &
            \multicolumn{1}{c}{割合} \\
            \hline
            \hline
            1 & 1 & 0.02\% \\
            2 & 1 & 0.02\% \\
            3 & 15 & 0.36\% \\
            4 & 57 & 1.36\% \\
            5 & 115 & 2.75\% \\
            6 & 259 & 6.20\% \\
            7 & 391 & 9.36\% \\
            8 & 568 & 13.60\% \\
            9 & 689 & 16.50\% \\
            10 & 634 & 15.18\% \\
            11 & 487 & 11.66\% \\
            12 & 267 & 6.39\% \\
            13 & 203 & 4.86\% \\
            14 & 126 & 3.02\% \\
            15 & 103 & 2.47\% \\
            16 & 67 & 1.60\% \\
            17 & 58 & 1.39\% \\
            18 & 42 & 1.01\% \\
            19 & 32 & 0.77\% \\
            20 & 26 & 0.62\% \\
            21 & 14 & 0.34\% \\
            22 & 6 & 0.14\% \\
            23 & 9 & 0.22\% \\
            24 & 2 & 0.05\% \\
            25 & 1 & 0.02\% \\
            26 & 1 & 0.02\% \\
            27 & 2 & 0.05\% \\
            29 & 1 & 0.02\% \\
            \hline
        \end{tabular}
    \end{table}

    \item[特徴量の数] 10
    \item[各特徴量の種類] \mbox{}
        元の特徴量は,性別,長さ,直径,高さ,全体重量,シャック重量,内臓重量,殻重量,年齢の9つであり,性別は,ワンホット変換されている.
    \item[留意事項] 提供元により,クラス1とそれ以外のクラスに2値化されている.
\end{description}


\subsection{sick\_euthyroid}
\begin{description}
    \item[概要] \cite{}
    \item[提供元] UCI Machine Learning Repository
    \item[サンプル数] 
    \item[クラスごとのサンプル数と割合] majority:  (\%), minority:  (\%) (詳細は表\ref{tab:}を参照)

        \begin{table}
            \centering
            \caption{のクラスごとのサンプル数}
            \label{tab:}
            \begin{tabular}{lrc} \hline
                \multicolumn{1}{c}{クラス}&
                \multicolumn{1}{c}{サンプル数}&
                \multicolumn{1}{c}{割合}\\
                \hline
                \hline

                \hline
            \end{tabular}
        \end{table}

    \item[特徴量の数] 
    \item[各特徴量の種類] \mbox{}
        
    \item[留意事項] 提供元により,クラス1とそれ以外のクラスに2値化されている.
\end{description}


\subsection{spectrometer}
\begin{description}
    \item[概要] \cite{}
    \item[提供元] UCI Machine Learning Repository
    \item[サンプル数] 
    \item[クラスごとのサンプル数と割合] majority:  (\%), minority:  (\%) (詳細は表\ref{tab:}を参照)

        \begin{table}
            \centering
            \caption{のクラスごとのサンプル数}
            \label{tab:}
            \begin{tabular}{lrc} \hline
                \multicolumn{1}{c}{クラス}&
                \multicolumn{1}{c}{サンプル数}&
                \multicolumn{1}{c}{割合}\\
                \hline
                \hline

                \hline
            \end{tabular}
        \end{table}

    \item[特徴量の数] 
    \item[各特徴量の種類] \mbox{}
        
    \item[留意事項] 提供元により,クラス1とそれ以外のクラスに2値化されている.
\end{description}

\subsection{car\_eval\_34}
\begin{description}
    \item[概要] \cite{}
    \item[提供元] UCI Machine Learning Repository
    \item[サンプル数] 
    \item[クラスごとのサンプル数と割合] majority:  (\%), minority:  (\%) (詳細は表\ref{tab:}を参照)

        \begin{table}
            \centering
            \caption{のクラスごとのサンプル数}
            \label{tab:}
            \begin{tabular}{lrc} \hline
                \multicolumn{1}{c}{クラス}&
                \multicolumn{1}{c}{サンプル数}&
                \multicolumn{1}{c}{割合}\\
                \hline
                \hline

                \hline
            \end{tabular}
        \end{table}

    \item[特徴量の数] 
    \item[各特徴量の種類] \mbox{}
        
    \item[留意事項] 提供元により,クラス1とそれ以外のクラスに2値化されている.
\end{description}

\subsection{isolet}
\begin{description}
    \item[概要] \cite{}
    \item[提供元] UCI Machine Learning Repository
    \item[サンプル数] 
    \item[クラスごとのサンプル数と割合] majority:  (\%), minority:  (\%) (詳細は表\ref{tab:}を参照)

        \begin{table}
            \centering
            \caption{のクラスごとのサンプル数}
            \label{tab:}
            \begin{tabular}{lrc} \hline
                \multicolumn{1}{c}{クラス}&
                \multicolumn{1}{c}{サンプル数}&
                \multicolumn{1}{c}{割合}\\
                \hline
                \hline

                \hline
            \end{tabular}
        \end{table}

    \item[特徴量の数] 
    \item[各特徴量の種類] \mbox{}
        
    \item[留意事項] 提供元により,クラス1とそれ以外のクラスに2値化されている.
\end{description}

\subsection{us\_crime}
\begin{description}
    \item[概要] 1990年の米国センサスデータ、1990年の米国LEMAS調査の法執行データ、1995年のFBI UCRの犯罪データに基づく米国内のコミュニティと犯罪に関するデータセット。\cite{us_crime}
    \item[提供元] UCI Machine Learning Repository
    \item[サンプル数] 1994
    \item[クラスごとのサンプル数と割合] majority:  (\%), minority:  (\%) (詳細は表\ref{tab:}を参照)

        \begin{table}
            \centering
            \caption{のクラスごとのサンプル数}
            \label{tab:}
            \begin{tabular}{lrc} \hline
                \multicolumn{1}{c}{クラス}&
                \multicolumn{1}{c}{サンプル数}&
                \multicolumn{1}{c}{割合}\\
                \hline
                \hline

                \hline
            \end{tabular}
        \end{table}

    \item[特徴量の数] 100
    \item[各特徴量の種類] \mbox{}
        人口統計,雇用統計,法執行データなど,様々な社会経済的指標を含む.例えば,人口,収入,失業率,警察の数,犯罪率など.
    \item[留意事項] 提供元により,クラス1とそれ以外のクラスに2値化されている.
\end{description}

\subsection{yeast\_ml8}
\begin{description}
    \item[概要] \cite{}
    \item[提供元] UCI Machine Learning Repository
    \item[サンプル数] 
    \item[クラスごとのサンプル数と割合] majority:  (\%), minority:  (\%) (詳細は表\ref{tab:}を参照)

        \begin{table}
            \centering
            \caption{のクラスごとのサンプル数}
            \label{tab:}
            \begin{tabular}{lrc} \hline
                \multicolumn{1}{c}{クラス}&
                \multicolumn{1}{c}{サンプル数}&
                \multicolumn{1}{c}{割合}\\
                \hline
                \hline

                \hline
            \end{tabular}
        \end{table}

    \item[特徴量の数] 
    \item[各特徴量の種類] \mbox{}
        
    \item[留意事項] 提供元により,クラス1とそれ以外のクラスに2値化されている.
\end{description}

\subsection{scene}
\begin{description}
    \item[概要] \cite{}
    \item[提供元] UCI Machine Learning Repository
    \item[サンプル数] 
    \item[クラスごとのサンプル数と割合] majority:  (\%), minority:  (\%) (詳細は表\ref{tab:}を参照)

        \begin{table}
            \centering
            \caption{のクラスごとのサンプル数}
            \label{tab:}
            \begin{tabular}{lrc} \hline
                \multicolumn{1}{c}{クラス}&
                \multicolumn{1}{c}{サンプル数}&
                \multicolumn{1}{c}{割合}\\
                \hline
                \hline

                \hline
            \end{tabular}
        \end{table}

    \item[特徴量の数] 
    \item[各特徴量の種類] \mbox{}
        
    \item[留意事項] 提供元により,クラス1とそれ以外のクラスに2値化されている.
\end{description}

\subsection{libras\_move}
\begin{description}
    \item[概要] \cite{}
    \item[提供元] UCI Machine Learning Repository
    \item[サンプル数] 
    \item[クラスごとのサンプル数と割合] majority:  (\%), minority:  (\%) (詳細は表\ref{tab:}を参照)

        \begin{table}
            \centering
            \caption{のクラスごとのサンプル数}
            \label{tab:}
            \begin{tabular}{lrc} \hline
                \multicolumn{1}{c}{クラス}&
                \multicolumn{1}{c}{サンプル数}&
                \multicolumn{1}{c}{割合}\\
                \hline
                \hline

                \hline
            \end{tabular}
        \end{table}

    \item[特徴量の数] 
    \item[各特徴量の種類] \mbox{}
        
    \item[留意事項] 提供元により,クラス1とそれ以外のクラスに2値化されている.
\end{description}

\subsection{thyroid\_sick}
\begin{description}
    \item[概要] \cite{}
    \item[提供元] UCI Machine Learning Repository
    \item[サンプル数] 
    \item[クラスごとのサンプル数と割合] majority:  (\%), minority:  (\%) (詳細は表\ref{tab:}を参照)

        \begin{table}
            \centering
            \caption{のクラスごとのサンプル数}
            \label{tab:}
            \begin{tabular}{lrc} \hline
                \multicolumn{1}{c}{クラス}&
                \multicolumn{1}{c}{サンプル数}&
                \multicolumn{1}{c}{割合}\\
                \hline
                \hline

                \hline
            \end{tabular}
        \end{table}

    \item[特徴量の数] 
    \item[各特徴量の種類] \mbox{}
        
    \item[留意事項] 提供元により,クラス1とそれ以外のクラスに2値化されている.
\end{description}

\subsection{coil\_2000}
\begin{description}
    \item[概要] \cite{}
    \item[提供元] UCI Machine Learning Repository
    \item[サンプル数] 
    \item[クラスごとのサンプル数と割合] majority:  (\%), minority:  (\%) (詳細は表\ref{tab:}を参照)

        \begin{table}
            \centering
            \caption{のクラスごとのサンプル数}
            \label{tab:}
            \begin{tabular}{lrc} \hline
                \multicolumn{1}{c}{クラス}&
                \multicolumn{1}{c}{サンプル数}&
                \multicolumn{1}{c}{割合}\\
                \hline
                \hline

                \hline
            \end{tabular}
        \end{table}

    \item[特徴量の数] 
    \item[各特徴量の種類] \mbox{}
        
    \item[留意事項] 提供元により,クラス1とそれ以外のクラスに2値化されている.
\end{description}

\subsection{arrhythmia}
\begin{description}
    \item[概要] \cite{}
    \item[提供元] UCI Machine Learning Repository
    \item[サンプル数] 
    \item[クラスごとのサンプル数と割合] majority:  (\%), minority:  (\%) (詳細は表\ref{tab:}を参照)

        \begin{table}
            \centering
            \caption{のクラスごとのサンプル数}
            \label{tab:}
            \begin{tabular}{lrc} \hline
                \multicolumn{1}{c}{クラス}&
                \multicolumn{1}{c}{サンプル数}&
                \multicolumn{1}{c}{割合}\\
                \hline
                \hline

                \hline
            \end{tabular}
        \end{table}

    \item[特徴量の数] 
    \item[各特徴量の種類] \mbox{}
        
    \item[留意事項] 提供元により,クラス1とそれ以外のクラスに2値化されている.
\end{description}

\subsection{solar\_flare\_m0}
\begin{description}
    \item[概要] \cite{}
    \item[提供元] UCI Machine Learning Repository
    \item[サンプル数] 
    \item[クラスごとのサンプル数と割合] majority:  (\%), minority:  (\%) (詳細は表\ref{tab:}を参照)

        \begin{table}
            \centering
            \caption{のクラスごとのサンプル数}
            \label{tab:}
            \begin{tabular}{lrc} \hline
                \multicolumn{1}{c}{クラス}&
                \multicolumn{1}{c}{サンプル数}&
                \multicolumn{1}{c}{割合}\\
                \hline
                \hline

                \hline
            \end{tabular}
        \end{table}

    \item[特徴量の数] 
    \item[各特徴量の種類] \mbox{}
        
    \item[留意事項] 提供元により,クラス1とそれ以外のクラスに2値化されている.
\end{description}

\subsection{oil}
\begin{description}
    \item[概要] \cite{}
    \item[提供元] UCI Machine Learning Repository
    \item[サンプル数] 
    \item[クラスごとのサンプル数と割合] majority:  (\%), minority:  (\%) (詳細は表\ref{tab:}を参照)

        \begin{table}
            \centering
            \caption{のクラスごとのサンプル数}
            \label{tab:}
            \begin{tabular}{lrc} \hline
                \multicolumn{1}{c}{クラス}&
                \multicolumn{1}{c}{サンプル数}&
                \multicolumn{1}{c}{割合}\\
                \hline
                \hline

                \hline
            \end{tabular}
        \end{table}

    \item[特徴量の数] 
    \item[各特徴量の種類] \mbox{}
        
    \item[留意事項] 提供元により,クラス1とそれ以外のクラスに2値化されている.
\end{description}

\subsection{car\_eval\_4}
\begin{description}
    \item[概要] \cite{}
    \item[提供元] UCI Machine Learning Repository
    \item[サンプル数] 
    \item[クラスごとのサンプル数と割合] majority:  (\%), minority:  (\%) (詳細は表\ref{tab:}を参照)

        \begin{table}
            \centering
            \caption{のクラスごとのサンプル数}
            \label{tab:}
            \begin{tabular}{lrc} \hline
                \multicolumn{1}{c}{クラス}&
                \multicolumn{1}{c}{サンプル数}&
                \multicolumn{1}{c}{割合}\\
                \hline
                \hline

                \hline
            \end{tabular}
        \end{table}

    \item[特徴量の数] 
    \item[各特徴量の種類] \mbox{}
        
    \item[留意事項] 提供元により,クラス1とそれ以外のクラスに2値化されている.
\end{description}

\subsection{wine\_quality}
\begin{description}
    \item[概要] \cite{}
    \item[提供元] UCI Machine Learning Repository
    \item[サンプル数] 
    \item[クラスごとのサンプル数と割合] majority:  (\%), minority:  (\%) (詳細は表\ref{tab:}を参照)

        \begin{table}
            \centering
            \caption{のクラスごとのサンプル数}
            \label{tab:}
            \begin{tabular}{lrc} \hline
                \multicolumn{1}{c}{クラス}&
                \multicolumn{1}{c}{サンプル数}&
                \multicolumn{1}{c}{割合}\\
                \hline
                \hline

                \hline
            \end{tabular}
        \end{table}

    \item[特徴量の数] 
    \item[各特徴量の種類] \mbox{}
        
    \item[留意事項] 提供元により,クラス1とそれ以外のクラスに2値化されている.
\end{description}


\subsection{letter\_img}
\begin{description}
    \item[概要] \cite{}
    \item[提供元] UCI Machine Learning Repository
    \item[サンプル数] 
    \item[クラスごとのサンプル数と割合] majority:  (\%), minority:  (\%) (詳細は表\ref{tab:}を参照)

        \begin{table}
            \centering
            \caption{のクラスごとのサンプル数}
            \label{tab:}
            \begin{tabular}{lrc} \hline
                \multicolumn{1}{c}{クラス}&
                \multicolumn{1}{c}{サンプル数}&
                \multicolumn{1}{c}{割合}\\
                \hline
                \hline

                \hline
            \end{tabular}
        \end{table}

    \item[特徴量の数] 
    \item[各特徴量の種類] \mbox{}
        
    \item[留意事項] 提供元により,クラス1とそれ以外のクラスに2値化されている.
\end{description}


\subsection{yeast\_me2}
\begin{description}
    \item[概要] \cite{}
    \item[提供元] UCI Machine Learning Repository
    \item[サンプル数] 
    \item[クラスごとのサンプル数と割合] majority:  (\%), minority:  (\%) (詳細は表\ref{tab:}を参照)

        \begin{table}
            \centering
            \caption{のクラスごとのサンプル数}
            \label{tab:}
            \begin{tabular}{lrc} \hline
                \multicolumn{1}{c}{クラス}&
                \multicolumn{1}{c}{サンプル数}&
                \multicolumn{1}{c}{割合}\\
                \hline
                \hline

                \hline
            \end{tabular}
        \end{table}

    \item[特徴量の数] 
    \item[各特徴量の種類] \mbox{}
        
    \item[留意事項] 提供元により,クラス1とそれ以外のクラスに2値化されている.
\end{description}


\subsection{webpage}
\begin{description}
    \item[概要] \cite{webpage}
    \item[提供元] UCI Machine Learning Repository
    \item[サンプル数] 34,780
    \item[クラスごとのサンプル数と割合] majority:  (\%), minority:  (\%) (詳細は表\ref{tab:}を参照)

        \begin{table}
            \centering
            \caption{webpageのクラスごとのサンプル数}
            \label{tab:webpage}
            \begin{tabular}{lrc} \hline
                \multicolumn{1}{c}{クラス}&
                \multicolumn{1}{c}{サンプル数}&
                \multicolumn{1}{c}{割合}\\
                \hline
                \hline

                \hline
            \end{tabular}
        \end{table}

    \item[特徴量の数] 300
    \item[各特徴量の種類] \mbox{}
        
    \item[留意事項] 提供元により,クラス1とそれ以外のクラスに2値化されている.
\end{description}


\subsection{ozone\_level}
\begin{description}
    \item[概要] \cite{}
    \item[提供元] UCI Machine Learning Repository
    \item[サンプル数] 
    \item[クラスごとのサンプル数と割合] majority:  (\%), minority:  (\%) (詳細は表\ref{tab:}を参照)

        \begin{table}
            \centering
            \caption{のクラスごとのサンプル数}
            \label{tab:}
            \begin{tabular}{lrc} \hline
                \multicolumn{1}{c}{クラス}&
                \multicolumn{1}{c}{サンプル数}&
                \multicolumn{1}{c}{割合}\\
                \hline
                \hline

                \hline
            \end{tabular}
        \end{table}

    \item[特徴量の数] 
    \item[各特徴量の種類] \mbox{}
        
    \item[留意事項] 提供元により,クラス1とそれ以外のクラスに2値化されている.
\end{description}


\subsection{mammography}
\begin{description}
    \item[概要] \cite{}
    \item[提供元] UCI Machine Learning Repository
    \item[サンプル数] 
    \item[クラスごとのサンプル数と割合] majority:  (\%), minority:  (\%) (詳細は表\ref{tab:}を参照)

        \begin{table}
            \centering
            \caption{のクラスごとのサンプル数}
            \label{tab:}
            \begin{tabular}{lrc} \hline
                \multicolumn{1}{c}{クラス}&
                \multicolumn{1}{c}{サンプル数}&
                \multicolumn{1}{c}{割合}\\
                \hline
                \hline

                \hline
            \end{tabular}
        \end{table}

    \item[特徴量の数] 
    \item[各特徴量の種類] \mbox{}
        
    \item[留意事項] 提供元により,クラス1とそれ以外のクラスに2値化されている.
\end{description}


\subsection{protein\_homo}
\begin{description}
    \item[概要] \cite{}
    \item[提供元] UCI Machine Learning Repository
    \item[サンプル数] 
    \item[クラスごとのサンプル数と割合] majority:  (\%), minority:  (\%) (詳細は表\ref{tab:}を参照)

        \begin{table}
            \centering
            \caption{のクラスごとのサンプル数}
            \label{tab:}
            \begin{tabular}{lrc} \hline
                \multicolumn{1}{c}{クラス}&
                \multicolumn{1}{c}{サンプル数}&
                \multicolumn{1}{c}{割合}\\
                \hline
                \hline

                \hline
            \end{tabular}
        \end{table}

    \item[特徴量の数] 
    \item[各特徴量の種類] \mbox{}
        
    \item[留意事項] 提供元により,クラス1とそれ以外のクラスに2値化されている.
\end{description}

\subsection{abalone\_19}

\begin{description}
    \item[概要] \cite{}
    \item[提供元] UCI Machine Learning Repository
    \item[サンプル数] 
    \item[クラスごとのサンプル数と割合] majority:  (\%), minority:  (\%) (詳細は表\ref{tab:}を参照)

        \begin{table}
            \centering
            \caption{のクラスごとのサンプル数}
            \label{tab:}
            \begin{tabular}{lrc} \hline
                \multicolumn{1}{c}{クラス}&
                \multicolumn{1}{c}{サンプル数}&
                \multicolumn{1}{c}{割合}\\
                \hline
                \hline

                \hline
            \end{tabular}
        \end{table}

    \item[特徴量の数] 
    \item[各特徴量の種類] \mbox{}
        
    \item[留意事項] 提供元により,クラス1とそれ以外のクラスに2値化されている.
\end{description}

\newpage