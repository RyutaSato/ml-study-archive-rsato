\section{データセット}
\label{sec:dataset}
本実験で使用する各データセットの詳細について述べる.
なお,クラス数はデータセットによって異なるため,2値分類問題は,データ数の多いクラスを`majority`クラスとして,他方を`minority`クラスとして集計する.
多クラスのデータセットにおいて,どのクラスを`minority`クラスとして評価するか,マクロ指標の定義については,別途定める.

\subsection{KDD CUP 1999 Data 10\%}
\begin{description}
    \item[概要] 1998年にアメリカの国防総省で開催されたKDD Cup 1999で使用されたデータセットの10\%データ.
    \item[サンプル数] 494,021
    \item[クラスごとのサンプル数] \mbox{}
        \begin{itemize}
            \item normal (majority):  97,278(19.7\%)
            \item dos:      391,458(79.2\%)
            \item probe:    4,628(0.9\%)
            \item r2l:      1,605(0.3\%)
            \item u2r (minority):      52(0.01\%)
        \end{itemize}
    \item[特徴量の数] 38 (カテゴリデータを除く)
    \item[各特徴量の種類] 正常アクセスと大きく4つに分類できる攻撃アクセスに関するインターネットトラフィックのデータが含まれている.
            duration, src\_bytes, dst\_bytes, land, wrong\_fragment, urgent, hot, num\_failed\_logins, logged\_in, num\_compromised, root\_shell, su\_attempted, num\_root, num\_file\_creations, num\_shells, num\_access\_files, num\_outbound\_cmds, is\_host\_login, is\_guest\_login, count, srv\_count, serror\_rate, srv\_serror\_rate, rerror\_rate, srv\_rerror\_rate, same\_srv\_rate, diff\_srv\_rate, srv\_diff\_host\_rate, dst\_host\_count, dst\_host\_srv\_count, dst\_host\_same\_srv\_rate, dst\_host\_diff\_srv\_rate, dst\_host\_same\_src\_port\_rate, \\dst\_host\_srv\_diff\_host\_rate, dst\_host\_serror\_rate, dst\_host\_srv\_serror\_rate, \\dst\_host\_rerror\_rate, dst\_host\_srv\_rerror\_rate
    \item[留意事項] このデータセットに含まれる3つのカテゴリデータは使用しない.
                   また,少数派クラスの評価にはu2rクラスのみを使用し,多数派クラスの評価にはnormalクラスのみを使用する.全体のmacro F1値は,これらのクラスを含めた5クラス全体で評価する.
\end{description}

\subsection{Credit Card Fraud Detection Dataset}
\begin{description}
    \item[概要] 金融取引データに基づくクレジットカード詐欺検出のためのデータセット.このデータセットは,正常な取引と不正な取引の2つのクラスに分類されたクレジットカード取引の情報を含んでいる.
    \item[提供元] 
    \item[サンプル数] 284,807
    \item[クラスごとのサンプル数と割合] \mbox{}
        \begin{itemize}
            \item 正常取引 (majority): 284,315(99.83\%)
            \item 詐欺取引 (minority): 492(0.17\%)
        \end{itemize}
    \item[特徴量の数] 特徴量の総数(例: 30)
    \item[各特徴量の種類] \mbox{}
            V1, V2, ..., V28(PCAによる匿名化された特徴量)、取引金額('Amount')、取引後の時間('Time')
    \item[留意事項] このデータセットには,主に数値特徴量が含まれており,PCAによって特徴量が匿名化されている.
\end{description}

\subsection{ecoli}
\begin{description}
    \item[概要] 大腸菌のタンパク質が細胞内のどの部分に局在するかを示すデータセット. \cite{ecoli}
    \item[提供元] UCI Machine Learning Repository
    \item[サンプル数] 336
    \item[クラスごとのサンプル数と割合] \mbox{}
        \begin{itemize}
            \item cp (majority): 143 (42.560\%)
            \item im (majority): 77 (22.917\%)
            \item imS (majority): 2 (0.595\%)
            \item imL (majority): 2 (0.595\%)
            \item imU (minority): 35 (10.417\%)
            \item om (majority): 20 (5.952\%)
            \item omL (majority): 5 (1.488\%)
            \item pp (majority): 52 (15.476\%)
        \end{itemize}
    \item[特徴量の数] 7
    \item[各特徴量の種類] 特徴量は,分析により判明した科学的特性を示したもの.\\
            mcg, gvh, lip, chg, aac, alm1, alm2
    \item[留意事項] 提供元により,`imU`クラスとそれ以外のクラスに2値化されている.
\end{description}

\subsection{optical\_digits}
    \begin{description}
    \item[概要] 手書きの数字の光学的認識を各ピクセルの値に基づいて示したデータセット.\cite{optical_digits}
    \item[提供元] UCI Machine Learning Repository
    \item[サンプル数] 5,620
    \item[クラスごとのサンプル数と割合] \mbox{}
        \begin{itemize}
            \item 8以外の数字 (majority): 5,010 (89.0\%)
            \item 8 (minority): 610 (11.0\%)
        \end{itemize}
    \item[特徴量の数] 64
    \item[各特徴量の種類] \mbox{}
        各特徴量は8$\times$8ピクセルのグリッドに基づく手書き数字の光学的特性を0から16の整数値で表したもの.
    \item[留意事項] 提供元により,`8`クラスとそれ以外のクラスに2値化されている.
    \end{description}

    \subsection{satimage}
    \begin{description}
        \item[概要] 衛星画像データを基にした土地利用分類のためのデータセット.\cite{satimage}
        \item[提供元] UCI Machine Learning Repository
        \item[サンプル数] 6,435
        \item[クラスごとのサンプル数と割合] \mbox{}
            \begin{itemize}
                \item 赤い土壌 (majority): 1,533 (23.8\%)
                \item 綿作物 (majority): 703 (10.9\%)
                \item 灰色の土壌 (majority): 1,358 (21.1\%)
                \item 湿った灰色の土壌 (minority): 626 (9.7\%)
                \item 切り株のある土壌 (majority): 707 (11.0\%)
                \item 非常に湿った灰色の土壌 (majority): 1508 (23.5\%)
            \end{itemize}
        \item[特徴量の数] 36
        \item[各特徴量の種類] \mbox{}
            各特徴量は,3$\times$3のピクセルそれぞれの4つのスペクトル (赤色,緑色,2種類の赤外線)の強さを0~255の値に変換したもの.
        \item[留意事項]  提供元により,`湿った灰色の土壌`クラスとそれ以外のクラスに2値化されている.
    \end{description}

\subsection{pen\_digits}
\begin{description}
    \item[概要] 44人の人物によって書かれた0から9までの数字のデータセット.\cite{pen_digits}
    \item[提供元] UCI Machine Learning Repository
    \item[サンプル数] 10,992
    \item[クラスごとのサンプル数と割合] \mbox{}
        \begin{itemize}
            \item 5 (minority): 1,057 (9.6\%)
            \item それ以外の数字 (majority): 9,935 (90.4\%) % TODO: データの数未確認
        \end{itemize}
    \item[特徴量の数] 16
    \item[各特徴量の種類] \mbox{}
        2次元座標上の描画における各点の座標を100msごとにサンプリングしたもの.
    \item[留意事項]  提供元により,`5`クラスとそれ以外のクラスに2値化されている.
\end{description}


