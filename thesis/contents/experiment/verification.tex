\section{検証内容}
本実験では,様々な設定を変え,精度への影響を調査する.
検証内容と比較する条件設定は以下の通りである.

\begin{description}
    \item [検証1] AE特徴量を加えることで,分類精度が向上するかどうかの検証
    \begin{itemize}
        \item 全てのデータを学習させたAutoEncoderから抽出したAE特徴量を加える場合
        \item AutoEncoderから抽出したAE特徴量を加えない場合
    \end{itemize}

    \item [検証2] AutoEncoderが学習するデータを特定のクラスに偏らせることが,分類精度に影響するかどうかの検証
    \begin{itemize}
        \item 全てのデータを学習させたAutoEncoderから抽出したAE特徴量を加える場合
        \item 多数派クラスのみを学習させたAutoEncoderから抽出したAE特徴量を加える場合
        \item 少数派クラスのみを学習させたAutoEncoderから抽出したAE特徴量を加える場合
    \end{itemize}

    \item [検証3] AutoEncoderの隠れ層の構成が,分類精度に影響するかどうかの検証
    \begin{itemize}
        \item Encoder層の次元数を[入力データの次元数, 20, 10, 5]とした場合
        \item Encoder層の次元数を[入力データの次元数, 20, 15, 10, 5]とした場合
    \end{itemize}

    \item [検証4] AE特徴量の数が,分類精度に影響するかどうかの検証
    \begin{itemize}
        \item AE特徴量の数を[入力データの次元数, 20, 10, 5]とした場合
        \item AE特徴量の数を[入力データの次元数, 20, 15, 10]とした場合
    \end{itemize}

    \item [検証5] 入力データの前処理によって,分類精度に影響するかどうかの検証
    \begin{itemize}
        \item 入力データの前処理を行わず,AutoEncoderに学習させた場合
        \item 標準化した入力データをAutoEncoderに学習させ,AE特徴量は標準化しない場合
        \item 標準化した入力データをAutoEncoderに学習させ,E特徴量も標準化する場合
        \item 正規化した入力データをAutoEncoderに学習させ,AE特徴量は正規化しない場合
        \item 正規化した入力データをAutoEncoderに学習させ,AE特徴量も正規化する場合
    \end{itemize}

    \item [検証6] オートエンコーダに学習させるデータを特定のクラス限定することによって,分類精度に影響するかどうかの検証
    \begin{itemize}
        \item 全てのデータを学習させたAutoEncoderから抽出したAE特徴量を加える場合
        \item 多数派クラスのみを学習させたAutoEncoderから抽出したAE特徴量を加える場合
        \item 少数派クラスのみを学習させたAutoEncoderから抽出したAE特徴量を加える場合
    \end{itemize}

    

\end{description}